% cc.tex
\documentclass[fleqn]{amsart}
\usepackage{hyperref}
\usepackage[utf8]{inputenc}
\usepackage{graphicx} % support the \includegraphics command and optio
%\usepackage[fleqn]{amsmath}
%\usepackage{amssymb}
%\usepackage{amsfonts}
%\usepackage{amsthm}
%\usepackage{authblk}

\newcommand{\Var}{\mathop{\rm{Var}}}
\newcommand{\Cov}{\mathop{\rm{Cov}}}
\theoremstyle{definition}
\newtheorem*{example}{}

\usepackage[parfill]{parskip} % Activate to begin paragraphs with an empty line rather than an indent

\title{Computing Cumulative Distributions using Cumulants}
\author{Keith A. Lewis}
\address{KALX, LLC}
\email{kal@kalx.net}
\date{\today}

\begin{document}
\maketitle
\begin{abstract}
This paper shows how to calculate cumulative distribution functions
of random variables using cumulants.
\end{abstract}

\section{Outline}

In section 2, we recall some basic facts about cumulants and their
relationship to moments involving Bell polynomials.  Section 3 uses an
Edgeworth expansion, and Hermite polynomials to get an explicit formula
for the cumulative distribution function of the perturbed distribution.

A formula for calculating the Esscher transform with unit argument is
developed in section 4 using the same technique as in the preceding
section.

The paper concludes with some general remarks.

\section{Cumulants}

The {\em cumulant generating function} of the random variable \(X\)
is \(\kappa^X(s) = \log Ee^{sX}\).
When they exist, the {\em cumulants}, \((\kappa_n)\),
are defined by
\[
\kappa(s) = \sum_{n=1}^\infty \kappa_n \frac{s^n}{n!}
\]
Since the \(n\)-th derivative evaluated at \(0\)
satisfies \(\kappa^{(n)}(s)|_{s = 0} = \kappa_n\) 
it is easy to
work out that
\(\kappa_1 = EX\), \(\kappa_2 = \Var(X)\). The third and fourth 
cumlants are related to skew and kurtosis. [Insert explicit formulas here.]

If \(X = c\) is constant then all cumulants are zero except \(\kappa_1 = c\).
I \(X\) is normally distributed with mean \(\mu\) and variance \(\sigma^2\)
then all cumulants are zero except \(\kappa_1 = \mu\) and
\(\kappa_2 = \sigma\). These are the only examples where the
cumulants are eventally 0. \cite{Luk1970} (Theorem 7.3.5).

The cumulants of a random variable plus a constant are the 
same except the first cumulant is increased by the constant.
More generally, the cumulants of the sum of two independent 
random variables are the sums of their cumulants.
They scale homogeneously, the \(n\)-th cumulant of a constant
times a random variable is
\(\kappa^{cX}_n = c^n\kappa^X_n\).

\section{Bell Polynomials}

The exponential of the cumulant in terms of
powers of \(s\) is
\begin{equation}
Ee^{sX} =  \exp(\sum_{n=1}^\infty \kappa_n \frac{s^n}{n!})
= \sum_{n=0}^\infty B_n(\kappa_1,\dots,\kappa_n) \frac{s^n}{n!}
\end{equation}\label{eq:1}
where \(B_n(\kappa_1,\dots,\kappa_n)\) is the \(n\)-th complete
Bell polynomial.
This is just a special case of the
Fa\`a di Bruno formula first proved by Louis Fran\c{c}ois Antoine
Arborgast in 1800\cite{Arb1800}.
Bell polynomials satisfy the recurrence \cite{Com1974} \(B_0 = 1\) and
\[
B_{n+1}(\kappa_1,\dots,\kappa_{n+1}) = \sum_{k=0}^n \binom{n}{k}
B_{n - k}(\kappa_1,\dots, \kappa_{n - k}) \kappa_{k+1}.
\]
This follows from (1) by differentiating with respect to \(s\) and equating powers of \(s\).
Note the second and third Bell polynomials are \(B_1(\kappa_1) = \kappa_1\), and
\(B_2(\kappa_1, \kappa_2) = \kappa_1^2 + \kappa_2\).
%$B_3(x_1, x_2, x_3) =
%= B2 x1 + 2 B1 x2 + B0 x3
%= x1^3 +  x2 x1 + 2 x1 x2 + x3
%x_1^3 + 3x_1x_2 + x_3$,
%B_4(x_1, x_2, x_4) =
%= B3 x1 + 3 B2 x2 + 3 B1 x3 + B0 x4
%= (x1^3 + 3 x_1 + x3)x1 + 3 (x1^2 + x2) x2 + 3 x1 x3 + B0 x4
%= (x1^4 + 3 x_1^2x_2 + x1x3) + 3 (x1^2x2 + x2^2)  + 3 x1 x3 + x4
%= x1^4 + 6 x_1^2x_2 + 4x1x3 + 3x2^2  + x4
%x1^4 + 6 x_1^2 x_2 + 4 x_1 x_3 + 3 x_2^2  + x4$.

\section{Edgeworth Expansion}
The standard normal random variable \(Z\) has density function \(\phi(x) =e^{-x^2/2}/\sqrt{2\pi}\). 
For any normal random variable \(N\) we have \(Ee^N = e^{EN + \Var(N)/2}\)
so the Fourier transform of \(Z\) is \(\hat{\phi}(u) = Ee^{-iuZ} = e^{-u^2/2}\).

Let $\psi$ be the probability density of a random variable, $X$,
having expected value 0 and mean 1.
The the Fourier transform is

\begin{align*}
\hat{\psi}(u) &= E e^{-iuX} \\
     &= e^{\sum_{n=1}^\infty \kappa_n (-iu)^n/n!}\\
     &= e^{-u^2/2} e^{\sum_{n=3}^\infty \kappa_n (-iu)^n/n!}\\
     &= e^{-u^2/2} \sum_{n=0}^\infty B_n(0, 0, \kappa_3,...,\kappa_n)(-iu)^n/n!.\\
	  &= e^{-u^2/2} (1 + \sum_{n=3}^\infty B_n(0, 0, \kappa_3,...,\kappa_n)(-iu)^n/n!).\\
\end{align*}

where \((\kappa_n)\) are the cumulants of \(X\).

The Fourier transform of \(f'\) is \(iu \hat f(u)\) so
the Fourier transform of the \(n\)-th derivative
\(f^{(n)}\) is \((iu)^n\hat f(u)\). Hence
\begin{align*}
\hat{\psi}(u) &= \hat{\phi}(u)(1 + \sum_{n=3}^\infty B_n(0, 0,\kappa_3,...,\kappa_n) (-iu)^n/n!)\\
	&= \hat{\phi}(u) + \sum_{n=3}^\infty (-1)^n B_n(0, 0, \kappa_3,...,\kappa_n)
	\widehat{\phi^{(n)}}(u)/n!\\
\end{align*}
Taking inverse Fourier transforms and integrating both
sides yields
\begin{equation}
	\Psi(x) = \Phi(x) + \sum_{n=3}^\infty (-1)^n B_n(0,0,\kappa_3,...,\kappa_n) \Phi^{(n)}(x)/n!.
\end{equation}

\subsection{Hermite Polynomials}
The derivatives of the standard normal density 
can be computed using Hermite polynomials\cite{AbrSte1964}
pp. 793--801.
One definition is
\[
H_n(x) = (-1)^n e^{x^2/2}\frac{d^n}{dx^n}e^{-x^2/2}.
\]
They satisfy the recurrence \(H_0(x) = 1\), \(H_1(x) = x\) and
\[
H_{n+1}(x) = xH_n(x) - n H_{n-1}(x).
\]
Note some authors use \(He_n(x)\) instead of \(H_n(x)\).
This shows \(\phi^{(n)}(x) = (-1)^n\phi(x) H_n(x)\)
so \(\Phi^{(n)}(x) = (-1)^{n-1} H_{n-1}(x)\) for \(n > 0\).

We now have an explicit formula for the cumulative
distibution function of \(X\):
\begin{equation}
\Psi(x) = \Phi(x) - \phi(x)\sum_{n=1}^\infty
B_n(\kappa_1,\dots,\kappa_n) H_{n-1}(x)/n!
\end{equation}

\section{The Esscher Transform}

Given any random variable, \(X\), and number \(s\), define
\(X^*\) by \(P(X^*\le x) = P^*(X\le x)\) where
\(dP^*/dP = e^{-\kappa(s) + sX}\).
The cumulant of \(X^*\) is \(\kappa^*(u) = \kappa(u + s) - \kappa(s)\),
where \(\kappa\) is the cumulant of \(X\).
This follows from \(E^* e^{uX} = Ee^{sX - \kappa(s)} e^{uX}
= e^{\kappa(s + u) - \kappa(s)}\).

It follows that the \(n\)-th derivative satisfies
\(\kappa^{*(n)}(u) = \kappa^{(n)}(u + s)\), for \(n > 0\),\
and setting $u=0$, \(\kappa^*_n = \kappa^{*(n)}(0) = \kappa^{(n)}(s)\).

The last expression can be expressed as \(\kappa^{(n)}(s) =
\sum_{k=0}^\infty \kappa_{n - k} s^k/k!\), but for many random variables,
we can use the closed form solution on the left-hand side instead of
the infinite sum.

%
%The Edgeworth series expands the exponential in a power series.
%\[\exp\bigl(\sum_{n=0}^\infty \Delta\kappa_n (iu)^n/n!\bigr)
%= \sum_{k=0}^\infty (\sum_{n=0}^\infty \Delta\kappa_n (iu)^n/n!)^k/k!\]
%

%\subsection{Computations}
%Explicit formulas the first few Bell polynomials:
%\begin{align*}
%B_0 &= 1\\
%B_1 &= x_1\\
%B_2 &= B_1 x_1 + B_0 x_2\\
%&= x_1^2 + x_2\\
%B_3 &= B_2 x_1 + 2B_1 x_2 + B_0 x_3\\
%&= x_1^3 + x_1 x_2 + 2x_1 x_2 + x_3\\
%&= x_1^3 + 3x_1 x_2 + x_3\\
%B_4 &= B_3 x_1 + 3 B_2 x_2 + 3 B_1 x_3 + B_0 x_4\\
%&= (x_1^3 + 3x_1 x_2 + x_3) x_1  + 3 (x_1^2 + x_2)x_2 + 3 x_1 x_3 + x_4\\
%&= x_1^4 + 6x_1^2 x_2 + 4 x_1 x_3 + 3x_2^2 + x_4\\
%\end{align*}
%Explicit formulas for the first few Hermite polynomials:
%\begin{align*}
%H_0 &= 1\\
%H_1 &= x\\
%H_2 &= x^2 - 1\\
%H_3 &= x^3 - x\\
%H_4 &= x^4 - 6x^2 + 3\\
%H_5 &= x^5 - 10x^3 + 15x\\
%H_6 &= x^6 - 15x^4 + 45x^2 - 15\\
%\end{align*}
%
%Assuming \(\kappa_1 = \kappa_2 = 0\) the first few terms of the Edgeworth expansion are:
%\begin{align*}
%G(x) &= \sum_{n=0}^\infty (-1)^n B_n F^{(n)}(x)/n!\\
%&= F(x) - B_1 F'(x) + B_2F''(x)/2 - B_3F^{(3)}(x)/6 + B_4F^{(4)}/24\\
%&= F(x) + (-\kappa_3 (x^2 - 1)/6 - \kappa_4 (x^3-x)/24)e^{-x^2/2}/\sqrt{2\pi}\\
%\end{align*}
%The distribution is unimodal if and only if the second derivative has exactly one root.
%\begin{align*}
%G''(x) &= \sum_{n=0}^\infty (-1)^n B_n F^{(n+1)}(x)/n!\\
%&= F^{(2)}(x) - B_1 F^{(3)}(x) + B_2F^{(4)}(x)/2 - B_3F^{(5)}(x)/6 + B_4F^{(6)}/24\\
%&= (-x - \kappa_3 (x^4 - 6x^2 + 3)/6
%	 - \kappa_4 (x^5 - 10x^3 + 15x)/24)e^{-x^2/2}/\sqrt{2\pi}\\
%\end{align*}
%
\subsection{Examples}
\begin{example}[Normal]
A normally distributed random variable, \(X\), has density function 
\(f(x) = e^{-(x - \mu)^2/2\sigma^2}/\sigma\sqrt{2\pi}\), \(-\infty<x<\infty\),
with mean \(\mu\) and variance \(\sigma^2\).
The cumulant is \(\kappa(s) = \mu s + \sigma^2s^2/2\) so
\(\kappa_1 = \mu\) and \(\kappa_2 = \sigma^2\) are the only non-zero
cumulants. 
\end{example}

Since \(\kappa^*(u) = \kappa(u + s) - \kappa(s) = (\mu + \sigma^2s)u + \sigma^2u/2\)
we see that \(\kappa^*_1 = \kappa_1 + \sigma^2\) and \(\kappa^*_2 = \kappa_2\). The
Esscher transform of a normally distributed random variable is normal
with mean \(\mu^* = \mu + \sigma^2\) and variance \(\sigma^{*2} = \sigma^2\).


\begin{example}[Gamma]
A gamma distributed random variable, \(X\), has density function
\(f(x) = x^{\alpha - 1} e^{-x/\beta}/\beta^\alpha\Gamma(\alpha)\), \(x > 0\),
with mean \(\alpha\beta\) and variance \(\alpha\beta^2\).
The cumulant is \(\kappa(s) = -\alpha\log(1 - \beta s)\) so
\(\kappa_n = \kappa^{(n)}(0) = (n-1)!\alpha\beta^n\) since
\(\kappa^{(n)}(s) = (n-1)!\alpha\beta^n/(1 - \beta s)^n\).

Since \(\kappa^*(u) = \kappa(u + s) - \kappa(s) = -\alpha\log (1 - \beta u/(1 - \beta s)\),
the Esscher transform of a gamma distributed random variable is gamma
with \(\alpha^* = \alpha\) and \(\beta^* = \beta/(1 - \beta s)\).

Note that an exponentially distributed random variable is the special case when 
\(\alpha = 1\).
\end{example}

\begin{example}[Poisson]
If \(X\) is Poisson
with mean \(\mu\) then \(\kappa(s) = \mu(e^s - 1)\) so
\(\kappa_n = \mu\) for all \(n\) and
\(\kappa_n^* = \mu e^s\).
\end{example}
%\begin{example}[Compound Poisson]
%If \(Y\) is Poisson with mean \(\mu\) and \(Z_j\) are
%independent and identically distributed, define
%\(X = \sum_{j=1}^{Y} Z_j\). The cumulants of \(X\)
%are \(\kappa_n = ?\).
%\end{example}
%\begin{align*}
%Ee^{uX} &= \sum_{k=0}^\infty Ee^{u(Z_1 + \cdots Z_k)}\mu^k/k!\,e^{-\mu}\\
%&= \sum_{k=0}^\infty (Ee^{u Z_1})^k\mu^k/k!\,e^{-\mu}\\
%&= \sum_{k=0}^\infty (Ee^{u Z_1}\mu)^k/k!\,e^{-\mu}\\
%&= e^{\mu Ee^{u Z_1}}\,e^{-\mu}\\
%&= e^{\mu (Ee^{u Z_1} - 1)}\\
%\end{align*}
%Define \(\lambda(u) = \log Ee^{uZ_1}\). Then
%\(\log E e^{uX} = \mu(e^{\lambda(u)} -1)\).

\begin{example}[VG]
The Variance Gamma model is the difference of independent Gamma distributions
so \(\kappa_n = (n-1)!(\alpha\beta^n - \alpha'\beta'^n)\).
In order for this to be a perturbation of a standard normal
distribution we need \(0 = \alpha\beta - \alpha'\beta'\)
and \(1 = \alpha\beta^2 - \alpha'\beta'^2\).
Using mean and standard deviation
\(\mu = \mu'\) and \(\sigma^2 = 1 + \sigma'^2\).
For convergence we need \(1 < \sigma^2 \ll \mu\).
\end{example}

\begin{example}[Compound Poisson]
Assume \(Y_i\) are independent and identically distributed. If \(N\) is
Poisson with parameter \(\lambda\) then \(X\) is {\em compound Poisson}
if \(X = \sum_{i=0}^N Y_i\).
The exponential of the cumulant of X is
\begin{align*}
Ee^{sX} &= Ee^{s\sum_{i=0}^N Y_i}\\
&= \sum_{k=0}^\infty \frac{\lambda^k}{k!}e^{-\lambda}(Ee^{Y_i})^k\\
&= \sum_{k=0}^\infty \frac{\lambda^k}{k!}e^{-\lambda}(e^{\kappa^Y(s)})^k\\
&= \sum_{k=0}^\infty \frac{(\lambda e^{\kappa^Y(s)})^k}{k!}e^{-\lambda}\\
&= e^{\lambda(e^{\kappa(s) - 1})}
\end{align*}
\end{example}

\begin{example}[Jump Diffusion]
Merton's \cite{Mer?} jump diffusion model assumes
\(X = \alpha Z + \beta\sum_{k=0}^N Y_k\) where
\(N\) is Poisson and \(Y_k\) are independent normal.
\end{example}

\begin{example}[Double Exponential]
\(f(y) = p\eta_1 e^{-\eta_1 y}1(y > 0) + (1 - p)\eta_2 e^{\eta_2 y}1(y < 0)\)

\begin{align*}
Ee^{sY} &= \int_0^\infty e^{sx} pe^{-\eta_1 y}\,dy
+ \int_{-\infty}^0 e^{sy} (1-p)e^{\eta_2 y}\,dy\\
&= \frac{p}{1-s/\eta_1} + \frac{1-p}{1 + s/\eta_2}
\end{align*}

\(\kappa^Y_n = n!(\frac{p}{\eta_1^n} + \frac{1-p}{(-\eta_2)^n})\)
\end{example}

%\begin{align*}
%Ee^{uX} &= \int_0^\infty e^{ux} x^{\alpha-1} e^{-\beta x}
%\frac{\beta^\alpha}{\Gamma(\alpha)}\,dx\\
%&= \int_0^\infty x^{\alpha-1} e^{-(\beta -u) x}
%\frac{\beta^\alpha}{\Gamma(\alpha)}\,dx\\
%&= \frac{\Gamma(\alpha)}{(\beta - u)^\alpha} \frac{\beta^\alpha}{\Gamma(\alpha)}\\
%&= (1 - u/\beta)^{-\alpha}\\
%\end{align*}
%so
%\(
%\log E e^{uX} = \alpha \sum_{n=1}^\infty (u/\beta)^n/n\).

%At the money pricing ???
%%F = fe^{-k(s) + s X}, x \le k(s)/s
%
%\begin{align*}
%&P(F\le f) - P^*(F\le f) \\
%&= P(Y\le \kappa(s)/s) - P^*(Y\le \kappa(s)/s) \\
%&= G(\kappa(s)/s) - G^*(\kappa(s)/s) \\
%&= \sum_{n=1}^\infty (-B_n(\Delta\kappa) + B_n(\Delta\kappa^*))
%    e^{-(\kappa(s)/s)^2/2} H_{n-1}(\kappa(s)/s)/sqrt{2\pi} \\
%& = \Phi(s/2) - \Phi(-s/2) \\
%\end{align*}

%\(\chi^2\) with \(r\) degrees of freedom \(\kappa_n = 2^{n-1}(n-1)!r\).


\subsection{Ininitely Divisible Random Variables}

%Let \(X = aN + bP + c\) where \(N\) is standard normal and \(P\)
%is Poisson with mean \(\mu\).

%\(EX = b\mu + c\) and \(\Var X = a^2 + b^2\). Taking
%\(b = \sqrt{1 - a^2}\)

Kolmogorov's precursor to the L\'evy-Khintchine theorem\cite{Kol1992}
states that if a random variable \(X\) is infinitely divisible
and has finite variance
there exists a number \(\gamma\) and a non-decreasing function
\(G\) defined on the real line such that
\[
\kappa(s) = \log Ee^{sX} = \gamma s + \int_{-\infty}^\infty K_s(x)\,dG(x),
\]
where \(K_s(x) = (e^{sx} - 1 - sx)/x^2 = \sum_{n=2}^\infty s^nx^{n-2}/n!\).
Note the first cumulant of \(X\) is \(\gamma\) and for \(n\ge 2\),
\(\kappa_n = \int_{-\infty}^\infty x^{n-2}\,dG(x)\).

The Hamburger moment problem\cite{ShoTam1943} provides the answer to
what the allowable cumulants are: the Hankel matrix
\([\kappa_{i+j}]_{i,j\ge 2}\) must be positive definite.

Since \(K_s(0) = s^2/2\) is the cumulant of the standard normal
distribution and \(a^2K_s(a) + as\) is the cumulant of a
Poisson distribution having mean \(a\),
infinitely divisible random variables can be
approximated by a normal plus a linear combination of
independent Poisson distributions.

%The K-model takes \(\gamma = 0\) and \(G\) of the form
%\[
%G(x) =
%\begin{cases}
%a e^{x/\alpha} &x < 0\\
%1 - be^{-x/\beta} & x > 0\\
%\end{cases}
%\]
%Note \(G\) jumps by \(1 - a - b\) at the origin. If \(a = b = 0\)
%this reduces to a standard normal distribution.
%
%The cumulants are simple to compute: \(\kappa_1 = 0\), \(\kappa_2 = 1\),
%and \(\kappa_{n+2} = (a(-\alpha)^n + b\beta^n)n!\) for \(n > 1\).


\section{Remarks}
The Gram-Charlier A series expands the quotients of cumulative
distribution functions \(G/F\) using Hermite polynomials, but does not
have asymptotic convergence, whereas the Edgeworth expansion involves
the quotient of characteristic functions \(\hat G/\hat F\) in terms of
cumulants and does have asymptotic convergence, ignoring some dainty
facts \cite{Pet1975}.

If \((X_t)\) is a L\'evy process then \(X_1\) is
infinitely divisible and \(\log Ee^{sX_t} = t\kappa(s)\).
A consequence is that the volatility smile at a single
maturity determines the entire volatility surface, a fact that
may indicate L\'evy processes are not appropriate for
modeling stock prices.

%A software implementation in C++ is available at
%\url{https://fmsgjr.codeplex.com} and Excel add-ins
%at \url{https://xllgjr.codeplex.com} that require
%\url{https://xll.codeplex.com}.

\bibliographystyle{amsplain}
\bibliography{njr}

\end{document}
